\documentclass[a4paper, 12pt]{article}
\usepackage[english]{babel}
\usepackage[utf8]{inputenc}
\usepackage[margin=1in]{geometry}
\usepackage{tcolorbox}
\usepackage{color, soul}

\begin{document}
\title{\textbf{Immunology} \\ \textbf{Innate Immune System}}
\author{Raymart Jay E. Canoy}
\date{\today}
\maketitle

\section{Overview}
\begin{itemize}
\item{Innate immune response}
\begin{itemize}
\item{Fast}
\item{Germline encoded}
\item{Limited specificity}
\item{Contains the infection}
\item{Activates the adaptive immune response}
\end{itemize}

\item{The innate immune system senses}
\begin{itemize}
\item{Foreign molecules}
\item{Host proteins modified by muicrobial enzymes}
\item{Changes caused by infections}
\item{Missing or stressed self}
\end{itemize}

\item{\hl{The innate immune system relies on four main types of receptors}}
\begin{itemize}
\item{Complement system}
\item{Fc receptors}
\item{Natural Killer cells receptors}
\item{Pattern recognition receptors}
\begin{itemize}
\item{Toll-like receptors}
\begin{itemize}
\item{TLR involved in \textcolor{red}{bacterial} PAMPs recognition}\\
(i) TLR2/1 and 2/6: Lipoprotein, \\
(ii) TLR4: Lipopolysaccharide, \\
(iii) TLR5: Flagelline, \\
(iv) TLR9: CpG DNA
\item{TLR involved in \textcolor{red}{viral and parasitic} PAMPs recognition} \\
(i) TLR2/1 and 2/6: Lipoprotein, \\
(ii) TLR3: double-stranded RNA \\
(iii) TLR7 and 8: single-stranded RNA\\
(iv) TLR9: CpG DNA
\end{itemize}
\end{itemize}
\end{itemize}

\item{The four key mechanisms that are involved in innate immune responses to microorganisms:}
\begin{enumerate}
\item{Phagocytosis}
\item{Intracellular killing}
\item{Cell recruitment}
\item{Antigen presentation}
\end{enumerate}
\end{itemize}

\section{Complement system}
\begin{itemize}
\item{Responds immediately and binds to molecular components of pathogens, particularly those that are already coated with antibodies}
\item{Successive complement proteins become activated in a sequential manner through:}
\begin{itemize}
\item{Cleavage (splitting), and/or}
\item{Structural changes leading to a series of enzymatic cascade events that result in pathogen desctruction}
\end{itemize}
\item{Derives its name from the fact that it is complementary to the antibody response of adaptive immunity.}
\item{\hl{Effector functions}}
\begin{enumerate}
\item{Opsonisation}
\item{Leukocyte activation}
\item{Cell lysis: Membrane attack complex}
\end{enumerate}

\item{Classical Pathway}
\begin{itemize}
\item{Triggered by binding of C1q to Fc portions of Immunoglobulin G (IgG) and IgM}
\item{CH2 domain of Fc of IgG3 > 1 > 2}
\item{CH3 domain of IgM}
\end{itemize}

\item{Lectin Pathway}
\begin{itemize}
\item{Mannose-binding lectin interacting with carbohydrate patterns}
\item{Ficolins interacting with acetyl patterns}
\item{Shares activation proteins with the classical pathway}
\item{C4b2a}
\end{itemize}

\item{Alternative pathway}
\begin{itemize}
\item{Constitutively active}
\item{C3bBb}
\item{C3b amplification}
\item{C3b is capable of cleaving C5}
\end{itemize}
\end{itemize}

\section{Microbiota}
\begin{itemize}
\item{\textcolor{red}{\textbf{Microbiota}} or commensal bacterial colonizes the majority of healthy epithelial tissues to keep pathogens under control.}
\begin{itemize}
\item{These species can produce antimicrobial molecules or trigger their production by the epithelial cells.}
\item{Microbiota homeostasis can be disturbed by different environmental factors or host genetics which leads to dysbiosis and can influence host inflammatory responses.}
\end{itemize}
\item{Functions of the innate immune system}
\begin{itemize}
\item{\textbf{Function 1}: Immediate, non-specific host defences}
\item{\textbf{Function 2}: Initiates appropriate adaptive responses}
\end{itemize}
\item{Mucosal epithelium}
\begin{itemize}
\item{\textcolor{red}{\textbf{Neutrophil}}}
\begin{itemize}
\item{Polymorphonuclear leukocyte}
\item{Not in mucosal tissue}
\item{\hl{Firs recruited by inflammation}}
\end{itemize}
\item{\textcolor{red}{\textbf{Macrophage}} and \textcolor{red}{\textbf{Dendritic cell}}}
\begin{itemize}
\item{Monuclear phagocytosis}
\item{Patrol mucosal tissue}
\item{Normally recruited after neutrophils}
\end{itemize}
\end{itemize}
\item{Cytokines produced by macrophage:}
\begin{itemize}
\item{Interleukin 1}
\begin{itemize}
\item{Vascular activation}
\item{Local tissue destruction}
\item{Increase access of immune cells}
\item{Systemic effect: fever}
\end{itemize}
\item{Tumour Necrosis Factor $\alpha$}
\begin{itemize}
\item{Vascular permeability}
\item{Increase access of immune cells}
\item{Systemic effect: fever}
\end{itemize}
\item{Interleukin 6}
\begin{itemize}
\item{Lymphocytes activation}
\item{Stimulates antibody production}
\item{Systemic effect: fever}
\end{itemize}
\item{Interleukin 8}
\begin{itemize}
\item{Chemotactic factor}
\item{Immune cells recruitment}
\end{itemize}
\end{itemize}
\item{The intestinal microbiota}
\begin{itemize}
\item{Humans have $~10^{14}$ intestinal bacteria}
\item{Provide nutrients and block pathogens}
\item{Many immune disorders connected to microbiota disruption}
\end{itemize}
\item{Disruption of the microbiota leads to}
\begin{itemize}
\item{Poorer responses to cancer treatment,}
\item{Type 1 diabetes,}
\item{Autoimmunity,}
\item{Susceptibility to infection}
\end{itemize}
\item{Germ-free animals are shown to have}
\begin{itemize}
\item{Smaller lymphoid structures,}
\item{Fewer dendritic cells and macrophages,}
\item{Fewer B cells and Th17 lymphocytes}
\end{itemize}
\item{\hl{The signals from the microbiota result to an enhanced immune response by the innate immune system.}}
\item{The commensal bacteria benefit from a nutritionally rich and protected habitat in the human GI tract, while they in turn benefit the host by making indigestible nutrients available to the body.}
\item{Some beneficial bacteria help restrict the access of pathogenic microorganisms to the gut tissue b building a protective biofilm.}
\item{\hl{The benefits of human microbe symbiosis can be extended to human mental health}}
\begin{itemize}
\item{The bidirectional communication between the resident microbes of the GI tract and the brain plays a key role in maintaining brain health.}
\item{\hl{The GI microbiota influences human behavior and may affect the pathophysiology of mental illnesses.}}
\item{The study of germ-free animals shows that brain development is abnormal  when the gut microbiome is missing. }
\item{The gut microbiome influences the inflammatory reactions within the brain by modulating the activation of microglial cells and affecting myelination and neurogenesis in adult brains.}
\item{\hl{Apart from the ENS, the vagus nerve is instrumental for the flow of information from the gut to the brain.}}
\end{itemize}
\end{itemize}

\section{Antigen processing - MCH Class I and Class II}
\begin{itemize}
\item{Innate immune responses are maintained in part by professional antigen presenting cells (APCs):}
\begin{itemize}
\item{Dendritic cells,}
\item{Monocytes/macrophages,}
\item{B cells}
\end{itemize}
\item{The different routes in which various antigen peptides (epitopes) can be processed and presented to a cell.}
\begin{enumerate}
\item{A pathogen or extracellular antigen is phagocytized by an antigen-presenting cell (dendritic cell) and placed into a vesicle. Ingested pathogens are digested by lysosomes to extract their antigens.}
\item{The antigens bind with MHC proteins that enter the vesicle.}
\item{The MHC proteins, now carrying antigens, are released from the vesible and travel to the outer surface of the cell membrane.}
\item{The dendritic cell is now presenting antigens, which will activate T cells that bind with the MHC proteins.}
\end{enumerate}
\item{Class II versus Class I}
\begin{itemize}
\item{Human Leukocyte Antigen II}
\begin{itemize}
\item{Expressed by specialised antigen presenting cells but expression can be induced on some other cell types.}
\item{Composed of $\alpha$ chain plus $\beta$ chain heterodimer}
\item{Presents antigenic peptides to CD4 helper cells}
\item{Has a peptide groove that can accommodate peptides of variable lengths, around 10-20 amino acids}
\item{Peptides are derived from uptake and processing of exogenous antigen, for example bacterial protein}
\item{Can present intracellular antigens}
\end{itemize}

\item{Human Leukocyte Antigen I}
\begin{itemize}
\item{Expressed by all nucleated cells (all cells, except red blood cells)}
\item{Composed of a single `heavy' chain plus $\beta$-2-microglobulin}
\item{Presents antigenic peptides to CD8 cytotoxic cells}
\item{Has a peptide groove with rather constrained peptide binding, accommodating peptides around 8-10 amino acids}
\item{Peptides are derived from endogenously transcribed antigens for example viral or tumour antigens}
\item{Can present exogenous peptides}
\end{itemize}
\end{itemize}
\end{itemize}

\section{Natural Killer Cells}
\begin{itemize}
\item{Natural killer cell in a nutshell}
\begin{itemize}
\item{Natural killer cells belong to group I Innate Lymphocytes (ILCs)}
\item{Responsible for a prompt response to a large variety of pathogenic microorganisms}
\item{Effective killers of virally infected cells and control early signs of cancer}
\item{Prior activation are not required for their activation}
\item{\textbf{Function:} These immune cells can recognise and kill the cells of someone's body that had been infected with a pathogen. Natural killer cells can also recognise and destroy tumour cells.}
\item{\textbf{Disease} People who have deficient natural killer cells, usually because of an inherited immune disorder, may be more prone to certain viruses.}
\item{\textbf{Location}: Natural killer cells (transcription factor: E4BP4) are present in the block and can move into other tissues to find targets.}
\end{itemize}
\item{\hl{Helper-like non-cytotoxic Innate Lymphocytes}}
\begin{itemize}
\item{There are three main groups of non-cytotoxic ILCs}
\begin{itemize}
\item{The groups are based on the cytokines they produce and the transcription factors that regulate their development and function}
\item{Many ILCs seem to reside at the body surfaces where they play a critical role in regulating epithelial cell responses and maintaining homeostasis.}
\item{ILC1 (T-bet) $\rightarrow$ Interferon-$\gamma$ and TNF$\alpha$: Protect against infections and can be found in the intestinal epithelium}
\item{ILC2 (ROR$\gamma$t) $\rightarrow$ IL5, IL13, and IL9: Respond to epithelial-derived cytokines, e.g., IL25 and IL33, by proliferating and producing IL5 and IL13}
\item{ILC3 (GATA3) $\rightarrow$ IL17 and IL22: Support lymphoid tissue development, help mucosal response via IL22, and support anti-bacterial defence via IFN$\gamma$}
\end{itemize}
\end{itemize}
\item{\hl{Natural killer cells}}
\begin{itemize}
\item{NK cell development primarily occurs within the bone marrow}
\item{Bone marrow ablation associated with dramatic defects in NK cell homeostasis and function}
\item{Third major lymphocyte subset: represent approximately 10-15\% of the peripheral-blood lymphocytes in humans}
\item{Potent lytic cells and cytokine producers}
\item{Part of the innate immune system so pivotal before the adaptive immunity develops}
\item{Act quickly as they do not need pre-sensitisation}
\item{Their activity is further boosted by cytokines, such as: IFN$\alpha$, IFN$\beta$, IFN$\gamma$, TNF$\alpha$, IL-12 and IL18}
\item{Main effector function}
\begin{itemize}
\item{Killing of target cells}
\item{Secretion of pro-inflammatory cytokines}
\end{itemize}
\item{Main effector mechanisms}
\begin{itemize}
\item{Use of stored cytotoxic proteins such as perforin and granzyme}
\item{Induce apoptosis through FASL and TRAIL}
\item{Secrete cytokines such as IFN$\gamma$ and TNF$\alpha$}
\end{itemize}
\item{Inhibitory NK receptors}
\begin{itemize}
\item{Recognise MHC class I molecules}
\item{Binding of MHC class I to the receptor leads to an inhibitory signal preventing the NK cell from killing a healthy cell}
\item{Inhibitory receptors carry distinct motifs called Immunoreceptor Tyrosine-based inhibitory motif (ITIM)}
\item{ITIM motifs recruit inhibitory tyrosine phosphatases SHP-1 and SHP-2 which leads to inhibition of signalling}
\end{itemize}
\item{Activating NK receptors}
\begin{itemize}
\item{Recognise either MHC class I or non-MHC class I molecules}
\item{Binding of an activating receptor in the absence of inhibitory signals leads to NK activation and cytotoxic activity}
\item{Some activating receptors can overcome inhibitory signals (NKG2D)}
\end{itemize}
\item{Two cell death pathways}
\begin{itemize}
\item{Perforin-dependent}
\begin{itemize}
\item{NK carry `lytic' or cytotoxic granules which are released when it interacts closely (forms a close synapse) and actively with a target cell}
\item{The main content of granules are perforin and several kinds of granzymes}
\item{Granzymes are serine proteases (enzymes) that cause DNA breakage and fragmentation}
\end{itemize}
\item{Death receptor-dependent}
\end{itemize}
\end{itemize}
\end{itemize}
\end{document}